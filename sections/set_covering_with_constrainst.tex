\section{Set covering with constraints}

For a family of sets $\cA$ its \emph{coverage} is $\cov\br{\cA}=\bigcup_{A\in\cA}A$, and by extension, $\cov\br{\cA, X} = \cov\br{\cA} \cap X$.
For $S \in \mathcal{G}$, let $\closure{S}$ denote the minimal precedence-closed subfamily of $\mathcal{G}$ containing $S$, i.e., $x\in\closure{S}$ if and only if $x\in S$ or $x\preceq y$ for some $y\in S$.
\DD{te definicje byc moze do przodu...}

The \emph{density} $\Delta$ of a nonempty subfamily $\cA$ on a universum $X$ ($A\subseteq X$ for each $A\in\cA$) is
\[
\Delta\br{\cA, X} \equiv \frac{\spr{\cov\br{\cA, X}}}{\spr{\cA}}
\]
For convenience, we define $\Delta\br{\emptyset, X} < 0$.
We also write $\Delta\br{\cA}$ when $X$ is the universum.
\begin{definition}[Max-Density Precedence-Closed Subfamily ($\ProblemMDPCS$)]
Given a family of $m$ sets $\mathcal{G}$, a precedence relation $\preceq$, and a set of $n$ items to be covered $R \subseteq \cov\br{\mathcal{G}}$, the $\ProblemMDPCS$ problem asks to find a precedence-closed subfamily $\cA \subseteq \mathcal{G}$ that maximizes $\Delta\br{\cA, R}$.
\end{definition}


\subsection{Precedence constrained set cover}
\newcommand{\cCopt}{\cC_{\textup{opt}}} %optimal solution to PCSC

In ourder to prove the main theorem of this section consider a greedy procedure shown as Algorithm~\ref{alg:PCSC}.
\begin{algorithm}[h]
\DontPrintSemicolon
\LinesNumbered
\caption{The $\gamma$-greedy algorithm for $\ProblemPCSC$}
\label{alg:PCSC}
\SetKwProg{Proc}{procedure}{}{}
\Proc{$\ProcedurePCSC(\cU, \cS, \preceq, k)$}{
$\cC \gets \emptyset$\;

\While{$\spr{\cov\br{\cC, \cU}} < k$}{
$\mathcal{A} \gets$ Run the $\gamma$-approx. algorithm for MDPCS on $(\cU \setminus \cov(\cC), \preceq, \cS\setminus\cC)$\;

% \If{$\spr{\cov\br{\cC\cup \mathcal{A}, \cU}} \geq k$}
% {
%     Find the minimum budget $B \in [\spr{\mathcal{A}}]$, such that the $\gamma$-approx. algorithm for MDPCS on $(\cU - \cC, \cS-\cC, \mathcal{F}-\cC, B)$ returns a set
%     $\mathcal{B}$ with $\spr{\cov\br{\mathcal{B}, \cU-\cC}} \geq \frac{k-\cov\br{\cC, \cU}}{\alpha}$\;
%
%     \While{$\spr{\cov\br{\cC, \cU}} < k$}{
%         $\mathcal{B} \gets$ Run the $\gamma$-approx. algorithm for MDPCS on $(\cU - \cC, \cS-\cC, \mathcal{F}-\cC, B)$\;
%
%         $\cC\gets \cC \cup \mathcal{B}$\;
%     }
%
%}

% \ForEach{$u \in \cov\br{\mathcal{A}, \cU-\cC}$}{
%     $c\br{u}\gets \Delta\br{\mathcal{A}, \cU-\cC}$\;
% }
$\cC \gets \cC \cup \mathcal{A}$\;
}
\Return $\cC$
}
\end{algorithm}

% \begin{algorithm}[h]
% \DontPrintSemicolon
% \LinesNumbered
% \caption{The $\gamma$-greedy algorithm for PCSC}
% \label{alg:PCSC}
% \SetKwProg{Proc}{procedure}{}{}
% \Proc{\textsc{PCSC}$(\cU, \cS, \mathcal{F}, K)$}{
% $\cC \gets \emptyset$\;
%
% \While{$\spr{\cov\br{\cC, \cU}} < K$}{
% $\mathcal{A} \gets$ Run the $\gamma$-approx. algorithm for MDPCS on $(\cU - \cC, \cS-\cC, \mathcal{F}-\cC, m)$\;
%
% \If{$\spr{\cov\br{\cC\cup \mathcal{A}, \cU}} \geq K$}
% {
%     Find the minimum budget $B \in [\spr{\mathcal{A}}]$, such that the $\gamma$-approx. algorithm for MDPCS on $(\cU - \cC, \cS-\cC, \mathcal{F}-\cC, B)$ returns a set
%     $\mathcal{B}$ with $\spr{\cov\br{\mathcal{B}, \cU-\cC}} \geq \frac{K-\cov\br{\cC, \cU}}{\alpha}$\;
%
%     \While{$\spr{\cov\br{\cC, \cU}} < K$}{
%         $\mathcal{B} \gets$ Run the $\gamma$-approx. algorithm for MDPCS on $(\cU - \cC, \cS-\cC, \mathcal{F}-\cC, B)$\;
%
%         $\cC\gets \cC \cup \mathcal{B}$\;
%     }
%
%     \Return $\cC$
% }
% \ForEach{$u \in \cov\br{\mathcal{A}, \cU-\cC}$}{
%     $c\br{u}\gets \Delta\br{\mathcal{A}, \cU-\cC}$\;
% }
% $\cC \gets \cC \cup \mathcal{A}$\;
% }
% }
% \end{algorithm}

Denote by $\cA_i$ the set $\cA$ from the $i$th iteration of Algorithm~\ref{alg:PCSC}, $i\in\brc{1,\ldots,l}$.
For brevity let $\cC_i$, $i\in\brc{1,\ldots,l}$, be the set $\cC$ obtained in the $i$th iteration, $\cC_i=\cA_1\cup\cdots\cup\cA_i$.
Let $R_i=\cU\setminus\cov\br{\cC_i}$ for $i\in\brc{1,\ldots,l}$ and $R_0=\cU$, $\cC_0=\emptyset$.
Denote by $\cI=(\cU,\cS,\preceq,k)$ any input instance to $\ProblemPCSC$, and let $\cCopt$ be an optimal solution to $\ProblemPCSC$ on $\cI$, i.e., $\cov\br{\cCopt}\geq k$ and $\spr{\cCopt}=\optPCSC{\cI}$.

\DD{tutaj mam zakomentowan jakas analize dla $l=1$, gdyz przez chwile myslalem, ze to da ostatecznie lepszy wynik}
% Consider first the case when $l=1$, i.e., there is only one iteration.
% Then, $\spr{\cov\br{\cA_l}}\geq k$ and $\Delta\br{\cA_l}\geq\gamma\cdot\Delta\br{\cCopt}$, where the latter follows from the assumption that the algorithm for $\ProblemMDPCS$ is $\gamma$-approximate.
% From this we obtain
% $\spr{\cA_l}  \leq \frac{1}{\gamma}\cdot \spr{\cov\br{\cA_l}}\cdot\spr{\cCopt}/k \leq \frac{n}{\gamma k}\spr{\cCopt}$, where the latter is by a trivial upper bound $\spr{\cov\br{\cA_l}}\leq n$.
Note that for each $i\in\brc{1,\ldots,l}$, $\cCopt\setminus\cC_{i-1}\neq\emptyset$ because otherwise $\cC_{i-1}\subseteq\cCopt$ which means that $\spr{\cC_{i-1}}\geq\spr{\cCopt}\leq k$ contradicting the fact that the algorithm conducted the $l$-th iteration.
Hence, $\cCopt\setminus\cC_{i-1}$ is a precedence closed family of density not greater than the density of an optimal solution to $\ProblemMDPCS$ for the input provided in the $i$-th iteration, $i\in\brc{1,\ldots,l}$.
Thus, $\Delta(\cA_i,R_{i-1}) \geq \gamma\cdot\Delta(\cCopt\setminus\cC_{i-1},R_{i-1})$, which gives by definition of $\Delta$,
$$
\frac{\spr{\cov\br{\cA_i,R_{i-1}}}}{\spr{\cA_i}} \geq \gamma\cdot\frac{\spr{\cov\br{\cCopt,R_{i-1}}}}{\spr{\cCopt\setminus\cC_{i-1}}}, \quad i\in\brc{1,\ldots,l}.
$$
$$
\spr{\cA_i} \leq \frac{1}{\gamma} \cdot \frac{ \spr{\cov\br{\cA_i,R_{i-1}}} \cdot \spr{\cCopt\setminus\cC_{i-1}} }{ \spr{\cov\br{\cCopt,R_{i-1}}} }
            \leq \frac{1}{\gamma} \cdot \frac{ \spr{\cov\br{\cA_i,R_{i-1}}} }{ \spr{\cov\br{\cCopt,R_{i-1}}} } \cdot \spr{\cCopt}.
$$
For each $i\in\brc{1,\ldots,l-1}$ it holds $\spr{\cov\br{\cCopt,R_{i-1}}}\geq k/2$.
Thus,
$$
\sum_{i=1}^{l-1}\spr{\cA_i} \leq \frac{2}{k\gamma} \cdot \spr{\cCopt} \cdot \sum_{i=1}^{l-1}\spr{\cov\br{\cA_i,R_{i-1}}} \leq \frac{2n}{k\gamma} \cdot \spr{\cCopt}.
$$
For the last iteration $\spr{\cov\br{\cA_l,R_{l-1}}}\leq k/2$ and $\spr{\cov\br{\cCopt,R_{l-1}}}\geq k/2$, we get $\spr{\cA_l}\leq \frac{1}{\gamma}\spr{\cCopt}$.
Hence, we obtain $\spr{\cC_l}=\sum_{i=1}^l\spr{\cA_i}\leq\frac{3n}{k\gamma}\cdot\spr{\cCopt}$.
Hence we obtain the following theorem.
\begin{theorem} \label{thm:MDPCStoPCSC}
    If there exists a $\gamma$-approximation algorithm for $\ProblemMDPCS$, then there exists a $(\frac{3n}{k\gamma},\frac{1}{2})$-approximation algorithm for $\ProblemPCSC$.
\end{theorem}

\MS{Tu poniżej miałem jakas próbe pisania tego, ale sie pokomplikowało więc ten pseuudokod jest niekompletny. To jest do zmiany wszystko. Zostawiam na razie te sekcje tobie Darku.}

We show the following:

\begin{theorem}
    If the precedence constraints form an outforest, then there exists an bicriteria $\br{4, O\br{\log n}}$-approximation algorithm for PCSC which be converted to an $O\br{\log^2 n}$ approximation algorithm.
\end{theorem}
\subsection{Precedence constrained min sum set cover}
By \cite{citehere}:
\begin{theorem}
    If there exists an $\gamma$ approximation algorithm for the $\ProblemMDPCS$ problem, then there exists an $4\cdot \gamma$ - approximate algorithm for the PCMSSC problem.
\end{theorem}
\begin{theorem}
    If the precedence constraints form an outforest, then there exists an $\br{O\br{\log n}}$-approximation algorithm for PCMSSC.
\end{theorem}


\subsection{Max-Density Precedence-Closed Subfamily (MDPCS)}

The key to solve PCSC and PCMSSC is to solve the $\ProblemMDPCS$ problem. An approximation algorithm for $\ProblemMDPCS$ can be used as an essential subroutine in our algorithms for PCSC and PCMSSC. By \cite{PCMSSC}, a greedy prodecure show in Algorithm~\ref{alg:MDPCS} achieves an $O\br{\sqrt{m}}$-approximation for $\ProblemMDPCS$.
\begin{algorithm}[h]
\DontPrintSemicolon
\LinesNumbered
\caption{The greedy algorithm for MDPCS}
\label{alg:MDPCS}
\SetKwProg{Proc}{procedure}{}{}
\Proc{\textsc{MDPCS-Greedy}$(\mathcal{G}, \prec, R)$}{
$\mathcal{A} \gets \mathcal{G}$\;

\ForEach{$S \in \mathcal{G}$}{
\If{$\Delta\br{P[S], R} > \Delta\br{\mathcal{A}, R}$}{
$\mathcal{A} \gets P[S]$\;
}
}
\Return $\mathcal{A}$\;
}
\end{algorithm}


Let $\delta = \max_{S \in \mathcal{G}} \Delta\br{\closure{S}, R}$. When $\delta\geq 1$, then the approximation factor of the greedy can also be bounded by $O\br{\sqrt{n}}$. In our application we will sometimes be interested in a version in which the elements of the universe can have arbitrary weight $w$ associated to them. Observe, that by scaling the weights up by a common denominator, we can transform weight to take integer values. By replacing each element $u \in \mathcal{U}$ with $w\br{u}$ representatives of $u$, we obtain an equivalent instance of unweighted MDPCS. Therefore, we observe that MDPCS-Greedy also achieves $O\br{\sqrt{m}}$and $\sqrt{w\br{\mathcal{U}}}$-approximation for the weighted MDPCS. 

We show that if we enforce a certain condition on the input called \emph{$\epsilon$-shallow ancestry}, then for $\epsilon <1$  the greedy algorithm achieves an $O\br{n^\epsilon}$-approximation. For each $S \in \mathcal{G}$, let $p(S) = \spr{\closure{S}}$ and $c(S, R) = \spr{\cov\br{\closure{S}, R}}$.
\begin{theorem}
Suppose there exists a constant $C > 0$ and $\epsilon \in (0, 1)$ such that for all $S \in \mathcal{G}$,
$
p(S) \leq C \cdot 
c(S)^\epsilon
$ ($\epsilon$-shallow ancestry).
Then $\ProblemMDPCS$-Greedy provides an $O(n^\epsilon)$-approximation.
\end{theorem}

\begin{proof}
 Let $\cA^*$ be an optimal solution consisting of sets $S_1, \ldots, S_k$. There are two cases:
\begin{enumerate}
    \item If $\delta \geq n^{1-\epsilon}$, then, we observe that $\Delta\br{\cA, R} = \delta \geq n^{1-\epsilon}$. Since we can cover at most $n$ elements with at least one set, we have $\Delta\br{\cA^*, R} \leq n$. Therefore:
$$
\frac{\Delta\br{\cA^*, R}}{\Delta\br{\cA, R}} \leq \frac{n}{n^{1-\epsilon}} = n^\epsilon
$$
    \item Else, if $\delta \leq n^{1-\epsilon}$, we proceed as follows: By definition of density, for any $S \in
     \mathcal{G}$, $c(S) \leq \delta \cdot p(S)$. Combining this with the $\epsilon$-shallow ancestry condition, we have that for all $S \in \mathcal{G}$, $
c(S) \leq \delta \cdot C \cdot c(S)^\epsilon
$. Rearraning this inequality, we get that $
c(S) \leq (\delta \cdot C)^{\frac{1}{1-\epsilon}}
$. We have that:
\[
\spr{\cov\br{\cA^*}} = \spr{\bigcup_{j=1}^k\cov\br{S_j, R}} \leq \sum_{j=1}^{k} c(S_j) \leq k \cdot (\delta \cdot C)^{\frac{1}{1-\epsilon}}
\]

Therefore:
\[
\Delta\br{\cA^*, R}= \frac{\spr{\cov\br{\cA^*, R}}}{k} \leq (\delta \cdot C)^{\frac{1}{1-\epsilon}}
\]

By the greedy choice, $\Delta\br{\cA, R} \geq \delta$ and by assumption $\delta \leq n^{1-\epsilon}$. Thus:
\[
\frac{\Delta\br{\cA^*, R}}{\Delta\br{\cA, R}} \leq \frac{(\delta \cdot C)^{\frac{1}{1-\epsilon}}}{\delta} = C^{\frac{1}{1-\epsilon}} \cdot \delta^{\frac{\epsilon}{1-\epsilon}} \leq C^{\frac{1}{1-\epsilon}} \cdot (n^{1-\epsilon})^{\frac{\epsilon}{1-\epsilon}} = C^{\frac{1}{1-\epsilon}} \cdot n^\epsilon
\]
\end{enumerate}
Since $C$ is constant, the theorem follows.
\end{proof}

