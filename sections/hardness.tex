\section{Hardness} \label{sec:hardness}
\newcommand{\coord}[3]{v[#1, #2, #3]}
\newcommand{\sepedge}[1]{e^{sep}[#1]}
\newcommand{\varedge}[1]{e^{var}[#1]}
\newcommand{\litedge}[1]{e^{cla}[#1]}

\subsection{Binary search with precedence constrains}

We begin by showing that the decision tree problem is NP-hard even for the very special of binary searching with precedence. In this problem we are given a linearly ordered set of hypotheses $h_1 \prec h_2 \prec \ldots \prec h_n$ and tests that correspond to binary queries of the form ``Is the unknown hypothesis $h_i$ less than or equal to $h_k$?''. Additionally, a precedence constraints on queries are given which enforce certain queries to be performed before others.

Interestingly, this problem is connected to a seemingly unrelated problem of parallel evaluation of arithmetic expressions. In this problem we are given an arithmetic expression consisting of $n$ variables combined using arbitrary binary operations. Additionally, due to the order of operations of the usage of brackets, a precedence constraints on procesing the operations is given. The goals is to evaluate this expression using arbitrary number of parallel processors however it is required that any operand can undergo only one operation at a time and the precedence constraints are satisfied. The goal is to minimize the total time required to evaluate the expression. It is widely known that without the precedence constraints this problem is equivalent to binary searching on a linearly ordered set of hypotheses. This connection is still valid even when precedence constraints are present, however the precedence constraints on the operations must be reversed. In what follows we show that the problem of binary searching with precedence is NP-hard thereby implying that parallel evaluation of arithmetic expressions with precedence constraints is NP-hard as well.
\begin{theorem}\label{thm:PCWCDT-NP-hard}
    Worst-Case Binary Search With Precedence Constraints is NP-hard.
\end{theorem}
\begin{proof}
	\TP{Bajzel}
	Consider an instance of \ProblemThreeSat given as: a set of variables $V$, and a set of clauses $C$.
	Each clause consist of exactly $3$ literals.
	Moreover, the variant where any $v \in V$ appears at most $5$ times as $v$ or $\overline{v}$ in the clauses is still \NPcomplete \cite{GareyAndJohnson}. 
	Obviously, in this variant $|C| = \cO(|V|)$.
	The idea is to:
	\begin{itemize}
		\item Divide the line into $2$ modules: variables and clauses. 
		Moreover, divide the modules into small segments, so each corresponds to a single variable, clause, respectively.
		Use precedence constraints to force this separation.
		\item All the variable vertices can be queried at the same time.
		Hence, either the variable or it's negation is quarried in a given turn.
		And the negation (or the variable) can be queried in the next.
		\item Make every literal dependent on the corresponding variable.
		\item Now, if at least one variable vertex was queried in the time $t$, then at least one literal vertex can be queried in $t+1$.
		\item If at least one literal vertex was queried then a clause gadget can be quarried in "short" time, because some "middle" vertex of the component was quarried. 
	\end{itemize}
	
	More formally, the mapping of an instance of \ProblemThreeSat given as a set of clauses, denoted by $C$, and variables, denoted by $V$, is as follows.
	To construct all other edges e graph:
	\begin{enumerate}
		\item Let $d > 0$ be smallest value such a value that $\log_2(|C| + |V| - 2 + d) \in Z$.
		Add $d$ dummy variables -- i.e. variables unrelated to any clause.
		From now, we assume that $|V| + |C| - 1 = 2^k-1$, for some $k > 0$.
		\item For every $c_i = \{v_i^1, v_i^2, v_i^3\}$, construct a gadget given as a path graph on vertices
		\[
			\coord{C}{i}{1}, \coord{C}{i}{v_i^1}, \coord{C}{i}{v_i^{12}}, \coord{C}{i}{v_i^{23}}, \coord{C}{i}{v_i^3}, \coord{C}{i}{2}.
		\]
		Denote the edges between the consecutive pairs of vertices as: $\litedge{i, 0}$, $\litedge{i, v_i^1}$, $\litedge{i, v_i^2}$, $\litedge{i, v_i^3}$, $\litedge{i, 0'}$.
		\item For a variable $v_i$, construct a gadget compose of $3$ vertices as a path
		\[
			\coord{v}{i}{1}, \coord{v}{i}{2}, \coord{v}{i}{3}
		\]
		Denote the edge $(\coord{v}{i}{1}, \coord{v}{i}{2})$ as $\varedge{v_i^-}$.
		Denote the edge $(\coord{V}{v}{2}, \coord{v}{i}{3})$ as $\varedge{v_i^+}$.
		\item Compose the graph in the following way.
		Connect each clause gadget one by one, let the vertex $\coord{C}{i}{4}$ is connected to $\coord{C}{i+1}{1}$ by and edge denoted as $\sepedge{i}$.
		Let us denote such an edge from this set as "separator edge".
		Connect $\coord{ M_{SAT}^*(I_S)C}{|C|}{4}$ to $\coord{V}{i}{1}$ by $\sepedge{|C|}$, $C$-th separator edge.
		Continue with variable gadgets in the same way; hence the total number of separator edges is $|C| + |V| - 1$.
	\end{enumerate}
	To define the order on the edges:
	\begin{enumerate}
		\item There are exactly $1+2+\ldots+2^{k}$ separator edges, for some $k > 0$.
		Hence easily one can make an order corresponding to binary tree, to force the queries.
		\item Force all other edges to be preceded by the separator edges.
		\item For every clause $C_i$ and every $v \in \{v_i^1,v_i^2,v_i^3\}$ make $\litedge{i, v}$ dependent on $\varedge{v^+}$ if $v$ is positive literal; make it dependent on $\varedge{v^-}$ otherwise.
		\item For every clause $C_i$, make $\litedge{i, 0}$ and $\litedge{i, 0'}$ dependent on all variable edges.
	\end{enumerate}
	Now, to prove that the problem is \NPhard we prove the decision version of the problem is \NPcomplete, 
	i.e. the problem to determine if for a given input $(P, O, t)$ there exists a strategy of maximum search time less than or equal to $t$.
	
	To prove that the problem is in \NP we observe that given search tree certifies maximum search time.
	
	Now, assume that the instance of $(C, V)$ \ProblemThreeSat is of 'YES' type.
	Hence let $f: V \leftarrow \{T, F\}$ be a valuation which satisfies all the clauses.
	Then, we build the strategy:
	\begin{itemize}
		\item Query the separator edges in time up to $k$.
		\item For every variable-vertex $v$, if variable was assigned $T$, query $\varedge{v^+}$ it in time $k+1$; otherwise query $\varedge{v^-}$.
		      Hence, they are queried in times $k+1$ and $k+2$, respectively. 
		\item Now all the clause gadgets have at least one precondition met in time $k+1$.
		      Due to this we can query at least one of $\litedge{i, v_i^1}$, $\litedge{i, v_i^2}$, $\litedge{i, v_i^3}$ in time $k+2$.
		      Due to the fact that for each clause there are up to $4$ candidate vertices, and each edge can be queried freely in time $k+3$ or later we can complete the queries in time up to $k+4$.
	\end{itemize}
	Due to this for the constructed graph there exists a strategy of time $k+4$.
	
	Assume now that the answer for the constructed instance, i.e. path graph, order, and time $k+4$ the answer is 'YES'.
	We perform the following observations.
	\begin{itemize}
		\item For a given clause-gadget corresponding, due to the fact that there are $5$ edges, we need at least $3$ time units to query it. 
		\item The earliest possible time to finish querying the separator edges is $k$.
		\item Due to this we have to perform at least one query in time $k+2$ for every clause-gadget $c_i$.
		Due to the fact that $\litedge{i, 0}$ and $\litedge{i, 0'}$ cannot be queried before $k+3$, it has to be the case that one of $\litedge{i, v_i^1}$, $\litedge{i, v_i^2}$, $\litedge{i, v_i^3}$ was queried.
		Hence, at for at least one of the edges preconditions were met (edge was queried) in time $k+1$.
		The variable-edges queried in $k+1$ to a valuation of the variables matching all the clauses.
	\end{itemize}
	
\end{proof}


\begin{theorem}
    Average-Case Binary Search With Precedence Constraints is \NPhard.
\end{theorem}
\begin{proof}
	The reduction is the same as in the worst-case variant; except that we have to consider the total query time.
	Again assume that the instance of \ProblemThreeSat is of 'YES' type.
	As previously, let $\log_2(|C| + |V|) = k$, hence $k \le 3 + \log_2m$.
	Hence:
	\begin{itemize}
	\item In any reasonable solution the separator-edges will have total quarry time $1 + 2*2 + 4*3 + 8*4 + \ldots 2^k*k$, denoted by $T_{sep}$.
	\item 	Now, the $|V|$ variable-edges correspond to total quarry time $|V| (2k+3)$, denoted by $T_{var}$.
	\item After this, consider the total quarry time of clause edges. 
	For a satisfied clause, there can be one quarry in time $k+2$ and other quarries $4$ can be done in time $k+3, k+3, k+4, k+4$. 
	Hence this gives total quarry time $|C|(5k + 16)$, denoted by $T_{clauses}$.  
	\end{itemize} 
	Together it gives the total quarry time $T_{sep} + T_{var} + T_{clauses}$ in a search tree corresponding to a valuation of variables satisfying all the clauses.
	The reverse direction, that a search tree of total quarry time $T_{sep} + T_{var} + T_{clauses}$ corresponds to a valuation satisfying all the clauses, is left to the reader.
	
\end{proof}
	
\begin{theorem}
		Worst-Case Tree Search With Precedence Constraints is cannot be approximated with ration better than $\frac{}{}$ unless $\Pclass = \NP$.
\end{theorem}
\begin{proof}
	However, to prove hardness we make an $L$-reduction from \ProblemMaxThreeSat.
	For the formal definition of the reduction we recommend \cite{Ausiello}.
	Namely, the problem where we would like to maximize the number of fulfilled clauses.
	To prove this we have to provide functions:
	\begin{itemize}
		\item $f$ -- mapping from the instances of \ProblemMaxThreeSat to Binary Search.
		\item $g$ -- mapping from the solutions of Binary Search to the solutions of \ProblemMaxThreeSat. 
	\end{itemize}
	And we need to prove the 2 crucial properties (other properties mentioned in the referenced definition are straightforward).
	For the first one, 
	let $I_S$ be any instance of \ProblemMaxThreeSat,
	let $f(I_S)$ -- the constructed instance of \ProblemAverageCaseBinarySearchPrecedanceConstraints,
	and let, $M_{SEARCH}$ and $M_{SAT}$ be the optimality measures for the respective problems (for the respective problems), also let $M_{SEARCH}^*$ and $M_{SEARCH}^*$ be the values of the optimal solutions for the instance.
	The property to prove is
	\[
	M_{SEARCH}^*(f(I_S)) \le \beta M_{SAT}^*(I_S)
	\]
	for some fixed $\beta > 0$.
	To prove this, we observe: 
	\begin{itemize}
		\item On one hand for any $I_S$ composed of $m$ clauses, we can observe at once that $M_{SAT}^*(I_S) \ge m/2$.
		\item On the other hand for such an instance, we can drop all variables which do not appear in any clause. 
		Due to this the number of variables is bounded by $3m$.
		Hence, for the instance, due to the dummy variables, $|C| + |V| \le 8m$.
		As previously, let $\log_2(|C| + |V|) = k$, hence $k \le 3 + \log_2m$.
		Hence, in an optimal solution the separator-edges will have total quarry time $1 + 2*2 + 4*3 + 8*4 + \ldots 2^k*k = C_{sep}$.
		\item Now, the $|V|$ variable-edges correspond to total quarry time $|V| (2k+3) = C_{var}$.
		\item After this, consider the total quarry time of clause edges. 
		For a satisfied clause, there can be one quarry in time $k+2$ and other quarries $4$ can be done in time $k+3, k+3, k+4, k+4$. 
		Hence this gives total quarry time $|C|(5k + 16)$ = $C_{clauses}$. 
	\end{itemize}
	Due to this if 
\end{proof}

\begin{theorem}
	Worst-Case Tree Search With Precedence Constraints is cannot be approximated with ration better than $\frac{}{}$ unless $\Pclass = \NP$.
\end{theorem}

\subsection{Outforest precendce constrains}
\begin{theorem}\label{thm:PCSC-to-PCWCDT-hardness}
    If there is no $\alpha$-approximation for PCSC, then there is no $\cO\br{\alpha}$-approximation for PCWCDT.
\end{theorem}
\begin{proof}
    Let $(U, \mathcal{S}, P)$ be an instance of PCSC where $U$ is the universe of elements, $\mathcal{S}$ is the family of sets and $P$ are the precedence constraints on $\mathcal{S}$. We construct an instance of PCWCDT as follows. For each element $u \in U$, we create two representative hypotheses $h_u^0$ and $h_u^1$. For each set $S \in \mathcal{S}$, we create a test $t_S$ such that $t_S(h_u^i) = h_u^i$ if $u \in S$ and $t_S(h_u^i) = 0$ otherwise. The precedence constraints on tests are the same as the precedence constraints on sets, i.e., if $S_1$ must be selected before $S_2$ in PCSC, then $t_{S_1}$ must be selected before $t_{S_2}$ in PCWCDT. Notice, that any valid decision tree for the constructed instance of PCWCDT is a path of tests (ignoring leafs) since performing any test $t_S$ either returns a response containing one of the hypotheses $h_u^i$, thus ending a process or returns $0$, which is a singular possible response leading to the next test. Note that even when the candidate set contains only representatives of one element $u$, we still need to perform a test to distinguish between $h_u^0$ and $h_u^1$, which could not be a case if there was only one hypothesis per element.
    Therefore, any valid decision tree for the constructed instance of PCWCDT corresponds to a valid selection of sets in PCSC and vice versa. Moreover, the cost of the decision tree is equal to the number of selected sets. Thus, since the size of the reduction is linear any $\alpha$-approximation for PCWCDT would yield an $\cO\br{\alpha}$-approximation for PCSC.
\end{proof}
\begin{theorem}\label{thm:PCMSSC-to-PCACDT-hardness}
    If there is no $\alpha$-approximation for PCMSSC, then there is no $\cO\br{\alpha}$-approximation for PCACDT.
\end{theorem}
\begin{proof}
    We use the same reduction as in the previous theorem. Note that since we doubled the number of hypotheses in the construction, the cost of any decision tree in the constructed instance of PCACDT is twice the number of selected sets in PCMSC. However, this does not affect the structure of the reduction and thus any $\alpha$-approximation for PCACDT would yield an $\cO\br{\alpha}$-approximation for PCMSSC.
\end{proof}
\subsection{Outforest precedence constraints}
\begin{theorem}\label{thm:PCWCDT-outforest-hardness}
    PCWCDT with outforest precedence constraints is NP-hard to approximate within a factor of $\cO\br{\log^{2-\epsilon} n}$ for any $\epsilon > 0$ unless $\text{NP}\subseteq \text{ZTIME}\br{n^{\text{polylog}(n)}}$.
\end{theorem}
\begin{proof}
    To show this we prove that Group Steiner Tree on tree metrics is reducible to Set Cover with outforest precedence constraints. Since GST on trees cannot be approximated within a factor of $\cO\br{\log^2 n}$ unless $\text{NP}\subseteq \text{ZTIME}\br{n^{\text{polylog}(n)}}$ \cite{PolylogarithmicInapproximability}. Note that in the latter reduction all of the weights are of form $2^{-h}$ with exponent ranging from $0$ to $h=\cO\br{\log^{1-\epsilon} n}$. Therefore, we can scale all weights by $2^h=\text{poly}\br{n}$ to obtain integer weights without affecting the approximation ratio. Let $T, w, \mathcal{G}$ be an instance of GST where $T$ is a tree metric with root $r$, $w$ are the weights on the edges of $T$ and $\mathcal{G}=\br{G_1, \ldots, G_k}$ are the groups. We construct an instance of PCSC with outforest precedence constraints as follows. Since to include a node in the Steiner Tree we need to include its parent edge of cost $w_e$, for each vertex $v\neq r$ we create its $w_e$ representatives $S_v^1 \preceq \ldots \preceq S_v^{w_e}$ so that taking $S^{w_e}$ to the cover requires taking all of the previous representatives. Additionally, for any directed edge $uv\in E$ such that $u \neq r$ and $e$ is a parent edge of $u$, we set $S^{w_e}_u \preceq S^1_v$ to enforce the condition that to including a node in a Steiner Tree requires including its parent node with its parent edge. For each group $G_i$ we create a universe element $u_i$. If $v \in G_i$ and $e$ is the parent edge of $v$, we set $S_{w_e}$ to cover $u_i$. It is easy to see that in order to cover any universe element by a vertex $v$ one needs to include all representatives of ancestors of $v$ excluding $r$. Therefore, any valid selection of sets in the constructed instance of PCSC corresponds to a valid Steiner Tree in the instance of GST and vice versa. Moreover, the cost of the selected sets is equal to the cost of the Steiner Tree. Thus, PCSC with outforest precedence constraints cannot be approximated within a factor of $\cO\br{\log^{2-\epsilon} n}$ for any $\epsilon > 0$ unless $\text{NP}\subseteq \text{ZTIME}\br{n^{\text{polylog}(n)}}$ and the same holds for PCWCDT with outforest precedence constraints.
\end{proof}
\subsection{General precedence constraints}
In this section we show strong inapproximability results for PCWCDT and PCACDT with general precedence constraints by reducing from the Planted Dense Subgraph Conjecture which is a widely believed statement about hardness of detecting a dense component within an Erdős-Renyi graph \cite{OnApproxTargetSetSelection,PCMSSC}.

The Planted Dense Subgraph Conjecture states that for any constants $ \beta < \alpha $ and any $k\geq \sqrt{N}$, there is no polynomial time algorithm that can distinguish between the following two distributions of graphs with any advantage $\epsilon > 0$:
\begin{itemize}
    \item With probability 1/2, $G_1$: an Erdős-Renyi graph $G(N, N^{\alpha - 1})$,
    \item With probability 1/2, $G_2$: an Erdős-Renyi graph $G(N, N^{\alpha - 1})$ with a planted subgraph of size $k$ and edge density $k^{\beta - 1}$.
\end{itemize}
Using this conjecture one can show the following inapproximability results for PCMSSC \cite{PCMSSC}:
\begin{theorem}\label{thm:PCMSSC-PDS-hardness}
    For any $\epsilon>0$ PCMSSC cannot be approximated within a factor of $\cO\br{m^{1/6-\epsilon}}$ nor $\cO\br{n^{1/12-\epsilon}}$ condition to Planted Dense Subgraph Conjecture.
\end{theorem}
Using the reduction from PCMSSC to PCACDT and PCWCDT from the previous section we immediately obtain the following results:
\begin{theorem}\label{thm:PCACDT-PDS-hardness}
    For any $\epsilon>0$ PCACDT cannot be approximated within a factor of $\cO\br{m^{1/6-\epsilon}}$ nor $\cO\br{n^{1/12-\epsilon}}$ condition to Planted Dense Subgraph Conjecture. 
\end{theorem}

Below we show that a similar reduction can also be used to obtain the same inapproximability result for PCSC. In the reduction we will often make use of the Chernoff Bound which is as follows:
\begin{theorem}[Chernoff Bound]\label{thm:chernoff}
    Let $X_1, X_2, \ldots, X_n$ be independent random variables taking values in $\brc{0,1}$ such that for every $i\in[n]$, $\mathbb{P}[X_i=1]=p$. Let $X=\sum_{i=1}^n X_i$ and $\mu=\mathbb{E}[X]=pn$. Then, for every $\delta\in(0,1)$ it holds that:
    \begin{align*}
        \mathbb{P}[X\leq (1-\delta)\mu] &\leq e^{-\frac{\delta^2\mu}{2}},\\
        \mathbb{P}[X\geq (1+\delta)\mu] &\leq e^{-\frac{\delta^2\mu}{3}}.
    \end{align*}
\end{theorem}

We have the following theorem:
\begin{theorem}\label{thm:PCSC-PDS-hardness}
    For any $\epsilon>0$ PCSC cannot be approximated within a factor of $\cO\br{m^{1/6-\epsilon}}$ nor $\cO\br{n^{1/12-\epsilon}}$ condition to Planted Dense Subgraph Conjecture.
\end{theorem}
\begin{proof}
    We start by choosing appropriate parameters for our reduction. Let $k=\sqrt{N}$, $\alpha=1/2$ and $\beta=1/2-\gamma$. In our reduction we will embded the structure of the graph into the precedence constraints, while the universe elements will merely enforce the covering requirement. Given a graph $G=\brc{[N], \mathcal{E}}$ we firstly convert each vertex $v$ of $G$ into $\lambda$ representative sets $V_{u, i}$ for $i \in [\lambda]$ where $\lambda$ is appropriately chosen natural number. We then convert each edge $uv\in\mathcal{E}$ into one representative set $E_{uv}$ which is a successor of all representatives of $u$ and $v$, i.e., $V_{u, i} \preceq E_{uv}$ and $V_{v, i} \preceq E_{uv}$ for all $i \in [\lambda]$. 

    We then create the universe elements $U=\brc{0,\dots, n}$ for some appropriately chosen natural number $n$. Every vertex representative $V_{u, i}$ covers consists of a singular item $V_{u, i}=\brc{0}$. In order to define the edge representatives we would firstly like to associate items with vertices. To do so we built auxiliary sets $U_v$ for each vertex $v$. The construction is randomized, and based on a probability $p\in(0,1)$. Each item in $[n]$ is associated with $U_v$ independently with probability $p$, so that $\mathbb{P}[j \in U_v] = p$ for each $j \in [n]$. Finally, each edge representative $E_{uv}$ covers the union of the auxiliary sets of its endpoints, i.e., $E_{uv} = U_u \cup U_v$. We have that $\mathbb{E}[\spr{U_v}] = p n$, $\mathbb{E}[\spr{v\colon j\in U_v}] = p N$ and $\mathbb{E}[\spr{E_{uv}}] = \mathbb{E}[\spr{U_u \cup U_v}] = np^2$.

    The idea is as follows: We pick $p$ to be smallest value for which if planted component exists then with high probability it covers $U$ in which case we can showcase a good solution to PCSC. On the other hand, if no planted component exists then we show that it pushes the cost of any solution to PCSC high enough to obtain the desired inapproximability ratio. 
    
    Let $\mathcal{P}$ be the planted component if it exists. Since $\mathcal{P}$ consists of $\sqrt{N}$ vertices we have that  $\mathbb{E}[\spr{v\colon j\in U_v, v \in V\brc{\mathcal{P}}}] = p\sqrt{N}$. Therefore, we also have that 
    $$
    \mathbb{E}[\spr{\brc{v, u}\colon j\in U_v, j\in U_u, v,u \in V\brc{\mathcal{P}}}] = \binom{p\sqrt{N}}
    {2}.$$
    
    Since each edge in $\mathcal{P}$ exists independently with probability $k^{\beta - 1} = \sqrt{N}^{-1/2 - \gamma}=N^{-1/4-\gamma/2}$ we have that $\mathbb{E}[\spr{E_{uv}\colon j\in E_{uv}}] = \binom{p\sqrt{N}}{2} \cdot N^{-1/4-\gamma/2}\geq p^2N/4 \cdot N^{-1/4-\gamma/2}=N^{3/4-\gamma/2}p^2$. So that this is large enough we pick $p=32\cdot N^{-3/8+\gamma/4}\cdot \log n$.

    Assume that the planted component exists. Then we show that by taking all edge representatives corresponding to edges in the planted component (and all of their predecessors) we can cheapily cover all items in $U$ with high probability. Since we have that $\mathbb{E}[\spr{v\colon j\in U_v, v \in V\brc{\mathcal{P}}}] = p\sqrt{N}$ by Chernoff's bound we have that:
    $$
    \mathbb{P}[\spr{v\colon j\in U_v, v \in V\brc{\mathcal{P}}} \leq p\sqrt{N}/2] \leq  e^{-p\sqrt{N}/8} = e^{-4\cdot N^{1/8+\gamma/4}\cdot \log n} = n^{-4\cdot N^{1/8+\gamma/4}}.
    $$

    which is extremely small. Since for large enough $x$, $\binom{x}{2} \geq x^2/4$ we have that:
    $$
    \mathbb{P}[\spr{uv\colon j\in U_v, j\in U_u, v,u \in V\brc{\mathcal{P}}} \leq p^2N/16] \leq e^{-4\cdot N^{1/8+\gamma/4}\cdot \log n} = n^{-4\cdot N^{1/8+\gamma/4}}.
    $$

    Since each edge in $\mathcal{P}$ exists independently with probability $N^{-1/4-\gamma/2}$ we have that:
    $$
    \mathbb{E}[\spr{E_{uv}\colon j\in E_{uv}, uv\in V\br{\mathcal{P}}}] \geq p^2N/16 \cdot N^{-1/4-\gamma/2}=4\cdot N^{3/4-\gamma/2}p^2=64\log^2 N.
    $$

    Let $\mu=\mathbb{E}[\spr{E_{uv}\colon j\in E_{uv}, uv\in V\br{\mathcal{P}}}]$. We have that:
    $$
    \mathbb{P}[\spr{E_{uv}\colon j\in E_{uv}, uv\in V\br{\mathcal{P}}} \leq \mu/2] \leq e^{-\mu/8} = e^{-8\log^2 N} = N^{-8\log N}.
    $$

    By applying the union bound, the probability that there exists an item $j\in[n]$ that is not covered by the representatives of the planted component is at most $n/N^{8\log N}$ which by our (future) choice of $n=\cO\br{N}$ is very small. Therefore, with high probability all items in $U$ are covered by taking all edge representatives corresponding to edges in the planted component (and all of their predecessors). Observe that:
    $$
    \mathbb{E}[\spr{E\br{\mathcal{P}}}]=\binom{\sqrt{N}}{2}\cdot N^{-1/4-\gamma/2}\leq N^{3/4-\gamma/2}/2.
    $$
    
    Again, using Chernoff's bound we have that:
    $$
    \mathbb{P}[\spr{E\br{\mathcal{P}}} \geq N^{3/4-\gamma/2}] \leq e^{-N^{3/4-\gamma/2}/6}.
    $$

    Therefore, with high probability the cost of the solution is at most $\lambda\sqrt{N}+N^{3/4-\gamma/2}$.
    Now assume that the planted component does not exist. We show that any solution to PCSC must have high cost with high probability. 

    
    \MS{Tutaj dowod jest jeszcze enigmatyczny dla mnie dlatego zostawiam to na razie. TODO: pokazac, ze jak nie ma komponentu to koszt rozwiazania mocno rosnie}
\end{proof}
Therefore we immediately have that:
\begin{theorem}\label{thm:PCWCDT-PDS-hardness}
    For any $\epsilon>0$ PCWCDT cannot be approximated within a factor of $\cO\br{m^{1/6-\epsilon}}$ nor $\cO\br{n^{1/12-\epsilon}}$ condition to Planted Dense Subgraph Conjecture. 
\end{theorem}
