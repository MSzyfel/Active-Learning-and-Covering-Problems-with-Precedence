\section{Hardness}
\newcommand{\coord}[3]{v[#1, #2, #3]}
\newcommand{\ecoord}[3]{e[#1, #2, #3]}
\newcommand{\sepedge}[1]{e^{sep}[#1]}
\newcommand{\varedge}[1]{e^{var}[#1]}
\newcommand{\litedge}[1]{e^{cla}[#1]}

\subsection{Binary search with precedence constrains}
\begin{theorem}
    Worst-Case Binary Search With Precedence Constraints is NP-hard.
\end{theorem}
\begin{proof}
	\TP{Bajzel}
	Consider an instance of \ProblemThreeSat given as: a set of variables $V$, and a set of clauses $C$.
	Each clause consist of exactly $3$ literals.
	Moreover, the variant where any $v \in V$ appears at most $5$ times as $v$ or $\overline{v}$ in the clauses is still \NPcomplete \cite{GareyAndJohnson}. 
	Obviously, in this variant $|C| = O(|V|)$.
	The idea is to:
	\begin{itemize}
		\item Divide the line into $2$ modules: variables and clauses. 
		Moreover, divide the modules into small segments so each corresponds to a single variable, clause, respectively.
		Use precedence constraints to force this separation.
		\item All the variable vertic1es can be queried at once.
		Hence, either the variable or it's negation is quarried in a given turn.
		And the negation (or the variable) can be queried in the next.
		\item Make every literal dependent on the corresponding variable.
		\item Now, if at least one variable vertex was queried in the time $t$, then at least one literal vertex can be queried in $t+1$.
		\item If at least one literal vertex was queried then a clause gadget can be quarried in "short" time, because some "middle" vertex of the component was quarried. 
	\end{itemize}
	
	More formally, the mapping of an instance of \ProblemThreeSat is as follows.
	To construct thMake all other edges e graph:
	\begin{enumerate}
		\item Let $d > 0$ be such a value that $\log_2(|C| + |V| + d) \in Z$.
		Add $d$ dummy variables -- i.e. variables unrelated to any clause.
		From now, we assume that $\log_2(|C| + |V|) \in Z$.
		\item For every $C_i = \{v_i^1, v_i^2, v_i^3\}$, construct a gadget given as a path graph on vertices
		\[
			\coord{C}{i}{1}, \coord{C}{i}{2}, \coord{C}{i}{v_i^1}, \coord{C}{i}{v_i^2}, \coord{C}{i}{v_i^3}, \coord{C}{i}{3}, \coord{C}{i}{4}.
		\]
		Denote the edge $(\coord{C}{i}{2}, \coord{C}{i}{v_i^1})$ as $\litedge{i, v_i^1}$.
		Denote the edge $(\coord{C}{i}{v_i^1}, \coord{C}{i}{v_i^2})$ as $\litedge{i, v_i^2}$.
		Denote the edge $(\coord{C}{i}{v_i^2}, \coord{C}{i}{v_i^3})$ as $\litedge{i, v_i^3}$.
		\item For a variable $v_i$, construct a gadget compose of $3$ vertices as a path
		\[
			\coord{v}{i}{1}, \coord{v}{i}{2}, \coord{v}{i}{3}
		\]
		Denote edge $(\coord{V}{i}{1}, \coord{V}{i}{2})$ as $\ecoord{V}{i}{+}$.
		Denote edge $(\coord{V}{i}{2}, \coord{V}{i}{3})$ as $\ecoord{V}{i}{-}$.
		\item Compose the graph in the following way:
		Connect each clause gadget one by one, by a "separator edge".
		I.e. the vertex $\coord{C}{i}{4}$ is connected to $\coord{C}{i+1}{1}$ by and edge denoted as $\sepedge{i}$
		Connect $\coord{C}{|C|}{4}$ to $\coord{V}{i}{1}$ by $\sepedge{|C|}$ $C$-th separator edge.
		Continue with variable gadgets in the same way.
	\end{enumerate}
	To define the order on the edges:
	\begin{enumerate}
		\item There are exactly $1+2+\ldots+2^{k}$ separator edges.
		Hence easily one can make an order corresponding to binary tree, to force the queries.
		\item Force all other edges to be preceded by the separator edges.
		\item For every clause $C_i$ and every $j \in \{1,2,3\}$ make $\litedge{i, v_i^j}$ dependent on $\varedge{v_i^j}$
	\end{enumerate}
	
\end{proof}

\begin{theorem}
    Average-Case Binary Search With Precedence Caseonstrains is NP-hard.
\end{theorem}
\subsection{Outforest precendce constrains}
\begin{theorem}
    PCWCAL with outforest precedence constraints is NP-hard to approximate within a factor of $O\br{\log^2 n}$ unless P = NP.
\end{theorem}
\subsection{General precedence constraints}
\begin{theorem}
    PCWCAL cannot be approximated within a factor of $o\br{m^{1/6}}$ nor $o\br{n^{1/12}}$ condition to Planted Dense Subgraph Conjecture. 
\end{theorem}
\begin{theorem}
    PCACAL cannot be approximated within a factor of $o\br{m^{1/6}}$ nor $o\br{n^{1/12}}$ condition to Planted Dense Subgraph Conjecture. 
\end{theorem}