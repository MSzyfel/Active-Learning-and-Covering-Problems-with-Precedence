\section{Binary search with precedence}
In this section, we consider the special case of the PCWCAL and PCACAL where the instance $I = (\mathcal{H}, \mathcal{T}, \mathcal{F})$ is an instance of the binary search problem with precedence constraints. In this setup we are given a linearly ordered set of $n$ elements $\mathcal{H} = \{h_1, h_2, \ldots, h_n\}$ with $h_1 \prec h_2 \prec \ldots \prec h_n$ and a set of tests $t_{i,j}\in \mathcal{T}$ corresponding to performing a comparison operation informing the learner whether the target element is less than or equal to $h_j$ or greater than $h_j$. This problem is NP-hard (see Section \ref{section}) however it is possible to derive an $O\br{\log n}$-approximation algorithm for both the PCWCAL and PCACAL.

The algorithm for a worst case is simple. We use the equivalence between binary searching in an ordered set and the edge ranking coloring of a path. An edge ranking of a path is a coloring of its edges such that any path between two edges of the same color contains an edge of a lower color \cite{citation}. Intuitevely, the color corresponds to the level of the decision tree where the test corresponding to the edge is performed. Let the input path be $P$. For any test $t\in\mathcal{T}$ define its \emph{depth} as $d\br{t}=\max_{\tau \in \mathcal{T}, P_{\tau, t}\in F}\brc{d\br{\tau, t}}+1$. Let the height of $\mathcal{F}$ be defined as $h\br{\mathcal{F}}=\max_{t\in \mathcal{T}}\brc{d\br{t}}$. The algorithm starts by partitioning $\mathcal{T}$ into $h\br{\mathcal{F}}$ sets $\mathcal{T}_1, \mathcal{T}_2, \ldots, \mathcal{T}_{h\br{\mathcal{F}}}$ such that for any $t\in \mathcal{T}_i$, $d\br{t}=i$. Then, the algorithm builds a decision tree processing layer one by one. Let $\mathcal{T}_i$ be such layer. We concatenate the (possibly disjoint) edges in $T_i$ into a path $P_i$. Then, we compute an optimal edge ranking coloring of $P_i$ using $\log n$ colors starting from color $\br{i-1}\cdot\log n$. Finally, we build a decision tree for the tests in $\mathcal{T}_i$ according to the edge ranking coloring. Observe that the resulting coloring is a valid edge ranking of $P$ and that each edge has a color greater than all its predecessors in $\mathcal{F}$. Thus, the precedence constraints are respected. The final decision tree $D$ is built recursively picking the root of the decision tree to be edge in $P$ with the smallest color. it is easy to see that $\COSTW\br{D, P}\leq h\br{\mathcal{F}}\cdot \log n$. Since the optimal decision tree has depth at least $h\br{\mathcal{F}}$, the algorithm is an $O\br{\log n}$-approximation.